\chapter{Manter Cliente - UC001} \label{uc001}
 
\section{Breve descrição}
 
O sistema permitirá o cadastro e a alteração dos dados do cliente.

\section{Atores}

\begin{enumerate}
	\item Funcionário da empresa, Casa das Marmitas.
\end{enumerate}

\section{Pré-condições}

\begin{enumerate}
	\item O funcionário deve estar logado no sistema.
	\item O funcionário deve ter permissão em seu login para realizar o cadastro e a alteração dos dados do cliente.
\end{enumerate}

\section{Fluxo de eventos}

\subsection{Fluxo básico}

\begin{enumerate}[label=P\arabic*]
	\item O funcionário aciona a opção \textbf{Cadastrar Cliente} no menu do sistema. \ref{uc001_a:1}
	\item O sistema apresenta a tela \textbf{Incluir Cliente} com os campos \ref{uc001_rn:1}. \label{uc001_p:2}
	\item O funcionário preenche os campos da tela e aciona a opção \textbf{Salvar}. \label{uc001_p:3} \ref{uc001_a:2}
	\item O sistema valida os dados dos campos. \ref{uc001_e:1} \ref{uc001_e:2} \ref{uc001_e:3} \ref{uc001_e:4} \ref{uc001_e:5} \ref{uc001_e:6} \ref{uc001_e:7}
	\item O sistema realiza a inclusão com sucesso.
	\item O sistema executa o caso de uso \nameref{uc013}.
	\item O caso de uso é encerrado.	
\end{enumerate}

\subsection{Fluxos alternativos}

\begin{enumerate}[label=A\arabic*]
	\item Alterar cliente \label{uc001_a:1}
	\begin{enumerate}[label*=.\arabic*]
		\item Na tela fornecida pelo sistema através do caso de uso \nameref{uc013}, o funcionário aciona a opção \textbf{Alterar}. 
		\item O sistema apresenta a tela \textbf{Alterar Cliente} com os campos \ref{uc001_rn:1}. \label{uc001_a:1:2}
		\item O funcionário preenche os campos da tela e aciona a opção \textbf{Salvar}. \label{uc001_a:1:3} \ref{uc001_a:2}
		\item O sistema valida os dados dos campos. \ref{uc001_e:1} \ref{uc001_e:2} \ref{uc001_e:3} \ref{uc001_e:4} \ref{uc001_e:5} \ref{uc001_e:6}
		\item O sistema altera os dados com sucesso.
		\item O sistema executa o caso de uso \nameref{uc013}.
		\item O caso de uso é encerrado.
	\end{enumerate}

	\item Cancelar inclusão ou alteração \label{uc001_a:2}
	\begin{enumerate}[label*=.\arabic*]
		\item No passo \ref{uc001_p:3} ou \ref{uc001_a:1:3}, o funcionário aciona a opção \textbf{Cancelar}.
		\item O sistema exibe a mensagem \textbf{Operação cancelada}.
		\item O sistema retorna ao passo anterior.
	\end{enumerate}

	\item Desistência
	\begin{enumerate}[label*=.\arabic*]
		\item O funcionário seleciona a opção \textbf{Voltar}.
		\item O sistema volta à tela anterior.
		\item O caso de uso é encerrado.		
	\end{enumerate}		 	
\end{enumerate}

\subsection{Exceções}

\begin{enumerate}[label=E\arabic*]
	\item O funcionário não informou algum campo obrigatório \label{uc001_e:1}
	\begin{enumerate}[label*=.\arabic*]
		\item[] No passo \ref{uc001_p:3} ou \ref{uc001_a:1:3}, o funcionário deixou em branco pelo menos um campo obrigatório.
		\item O sistema exibe a mensagem \textbf{Favor preencher o campo obrigatório}.
		\item O sistema destaca os campos não informados pelo funcionário.
		\item O sistema retorna ao passo anterior.
	\end{enumerate}

	\item O funcionário preencheu de forma errada o campo da data de nascimento \label{uc001_e:2}
	\begin{enumerate}[label*=.\arabic*]
			\item[] No passo \ref{uc001_p:3} ou \ref{uc001_a:1:3}, o funcionário não preencheu de forma correta o campo da data de nascimento, por exemplo, usou somente letras.				
		\item O sistema exibe a mensagem \textbf{A data de nascimento informada não é válida}.
		\item O sistema destaca o campo \textbf{Data de nascimento}.
		\item O sistema retorna ao passo anterior.
	\end{enumerate}

	\item O funcionário preencheu de forma errada o campo de telefone \label{uc001_e:3}
	\begin{enumerate}[label*=.\arabic*]		
		\item[] No passo \ref{uc001_p:3} ou \ref{uc001_a:1:3}, o funcionário não preencheu de forma correta o campo de telefone, usando letras ou qualquer outro carácter diferente de número.		
		\item O sistema exibe a mensagem \textbf{O número de telefone informado não é válido}.
		\item O sistema destaca o campo \textbf{Telefone}.
		\item O sistema retorna ao passo anterior.
	\end{enumerate}

	\item O funcionário preencheu de forma errada o campo de logradouro \label{uc001_e:4}
	\begin{enumerate}[label*=.\arabic*]		
		\item[] No passo \ref{uc001_p:3} ou \ref{uc001_a:1:3}, o funcionário não preencheu de forma correta o campo de logradouro (rua, avenida, beco, ...), por exemplo, usou somente números.		
		\item O sistema exibe a mensagem \textbf{O nome do logradouro informado não é válido}.
		\item O sistema destaca o campo \textbf{Logradouro}.
		\item O sistema retorna ao passo anterior.
	\end{enumerate}

	\item O funcionário preencheu de forma errada o campo de CEP \label{uc001_e:5}
	\begin{enumerate}[label*=.\arabic*]		
		\item[] No passo \ref{uc001_p:3} ou \ref{uc001_a:1:3}, o funcionário não preencheu de forma correta o campo de CEP, usando letras ou menos de 8 dígitos.		
		\item O sistema exibe a mensagem \textbf{O CEP informado não é válido}.
		\item O sistema destaca o campo \textbf{CEP}.
		\item O sistema retorna ao passo anterior.
	\end{enumerate}

	\item O funcionário preencheu de forma errada o campo de cidade \label{uc001_e:6}
	\begin{enumerate}[label*=.\arabic*]		
		\item[] No passo \ref{uc001_p:3} ou \ref{uc001_a:1:3}, o funcionário não preencheu de forma correta o campo de cidade, por exemplo, usou somente números.		
		\item O sistema exibe a mensagem \textbf{O nome da cidade informado não é válido}.
		\item O sistema destaca o campo \textbf{Cidade}.
		\item O sistema retorna ao passo anterior.
	\end{enumerate}

	\item Cliente cadastrado anteriormente \label{uc001_e:7}
	\begin{enumerate}[label*=.\arabic*]
		\item[] No passo \ref{uc001_p:3} do fluxo básico, após acionamento da opção \textbf{Salvar} pelo funcionário.
		\item O sistema exibe a mensagem \textbf{O cliente informado já foi cadastrado anteriormente no sistema}.
		\item O sistema retorna ao passo anterior.
	\end{enumerate}
\end{enumerate}

\section{Pós-condições}

\begin{enumerate}
	\item Cliente cadastrado
	\begin{enumerate}
		\item O cliente estará, devidamente, cadastrado no banco de dados.
		\item O sistema carregará a tela com os dados do cliente.
	\end{enumerate}

	\item Cliente alterado
	\begin{enumerate}
		\item Os dados do cliente estarão, devidamente, atualizados no banco de dados.
		\item O sistema carregará a tela com os dados do cliente.
	\end{enumerate}
\end{enumerate}

\section{Regras de negócios especiais}

\begin{enumerate}[label=RN\arabic*]
	\item Campos apresentados na tela de inclusão e alteração dos dados do cliente deverão esta de acordo com a tabela \ref{uc001_tb_rn1}. \label{uc001_rn:1}
	\begin{table}[htb]
		\ABNTEXfontereduzida
		\caption[Dados do cliente]{Dados do cliente.}
		\label{uc001_tb_rn1}
		\begin{tabular}{|p{3.0cm}|p{2.0cm}|p{1.5cm}|p{2.0cm}|p{5.75cm}|}
			\hline
			\textbf{Campo}      & \textbf{Tipo} & \textbf{Tamanho} & \textbf{Obrigatório} & \textbf{Observação}                                                                                                                                \\ \hline
			Cód. do cliente     & Número        & 18               & SIM                  & Atributo identificador, chave primária, responsável por identificar o id da tabela de clientes.                                                    \\ \hline
			Data cadastro       & dd/mm/aaaa    & 10               & SIM                  & Atributo simples da tabela de clientes responsável por armazenar, no banco e dados, a data em que o cliente foi cadastrado no sistema.                \\ \hline
			Nome                & Texto         & 60               & SIM                  & Atributo simples da tabela de clientes responsável por armazenar, no banco e dados, o nome do cliente.                                             \\ \hline
			Data de nascimento  & dd/mm/aaaa    & 10               & NÃO                  & Atributo simples da tabela de clientes responsável por armazenar, no banco e dados,  a data de nascimento do cliente.                              \\ \hline
			Telefone            & Número        & 10               & SIM                  & Atributo simples da tabela de clientes responsável por armazenar, no banco e dados, o número de telefone do cliente.                               \\ \hline
			Logradouro          & Texto         & 100              & SIM                  & Atributo simples da tabela de clientes responsável por armazenar, no banco e dados, o logradouro do cliente.                                       \\ \hline
			CEP                 & Número        & 10               & SIM                  & Atributo simples da tabela de clientes responsável por armazenar, no banco e dados, o CEP do cliente.                                              \\ \hline
			Bairro              & Texto         & 60               & SIM                  & Atributo simples da tabela de clientes responsável por armazenar, no banco e dados, o bairro no qual o cliente está situado.                       \\ \hline
			Cidade              & Texto         & 60               & SIM                  & Atributo simples da tabela de clientes responsável por armazenar, no banco e dados, a cidade na qual o cliente esta situado.                       \\ \hline
			Número              & Texo          & 30               & SIM                  & Atributo simples da tabela de clientes responsável por armazenar, no banco e dados, o numero da residência do cliente.                             \\ \hline
			Complemento         & Texto         & 30               & NÃO                  & Atributo simples da tabela de clientes responsável por armazenar, no banco e dados, alguma informação relevante em relação ao endereço do cliente. \\ \hline
			Ponto de referência & Texto         & 50               & NÃO                  & Atributo simples da tabela de clientes responsável por armazenar, no banco e dados, algum ponto de referência próximo à casa do cliente            \\ \hline
		\end{tabular}
	\end{table}
\end{enumerate}